
\hspace{0pt}
\vfill

\invisiblesection{Portada}

\begin{center}
    {\huge Aplicación del Álgebra Lineal:\\Ofuscación binaria}\\    \quad\\
    {\large Universidad Panamericana}\\
    {\large Facultad de Ingeniería}\\
    \quad\\
    \quad\\
    \includegraphics[scale=0.3]{../img/UP}
    \quad\\
    \quad\\
    Inteligencia Artificial\\
    Dr. Ari Yair Barrera Animas\\
    \quad\\
    \quad\\
    \begin{tabular}{c|c}
        Osornio López Daniel Alejandro & 0244685\\
		Daniel Hernandez Toledo & 
    \end{tabular}\\
    \quad\\
    \quad\\
	02 22 2022
\end{center}

\vfill

\newpage
\tableofcontents

\newpage
\section{Objetivos}

1. Design and implement searching algorithms for graph structures using
the Rust programming language

\begin{enumerate}[label={}]
\item 1.1 Breadth First
\item 1.2 Uniform Cost (Dijkstra)
\item 1.3 Depth First
\item 1.4 Limited Depth First
\item 1.5 Iterative Depth First
\item 1.6 Bidirectional
\end{enumerate}



2. Correct and efficient documentation of the code and features


\section{Resumen}

El objetivo de este proyecto es utilizar transformaciones lineales como llaves
para cifrar y descifrar bits de información.
Se explora el uso de la aritmética modular para mantener la congruencia de la
información
 a la vez que se experimenta con distintos métodos de cifrado por medio de
algoritmos varios que emplean las transformaciones en su proceso.
La meta es implementar una herramienta de cifrado y descifrado que pueda
proteger tanto archivos de texto como ejecutables con un procedimiento
que permita su auto descifrado junto con soporte a \textit{passphrases} para
obtener mayor seguridad.

Los métodos estudiados son dos. 

El primero es una modificación de un cifrado de sustitución. Primero se
construye una matriz que contiene todos los valores representables en un 
byte y modifica la matriz por medio de multiplicaciones matriciales para
después sustituir cada byte en el archivo por su nuevo valor. 

El segundo, ordena todos los bytes del mensaje a cifrar, como puede ser un
archivo, y le aplica una multiplicación matricial a la totalidad del archivo. 

Para dar soporte al auto-descifrado de los archivos, cuando se utilice la
bandera --auto, se almacenan en el archivo los valores para descifrarse a sí
mismo, 
con bytes de meta información que facilitan el proceso y permiten una mayor
seguridad lograda por el factor aleatorio. Este proyecto plantea la
implementación 
de una librería de cifrado de archivos por medio de una ofuscación binaria
lograda por transformaciones.

\newpage
\section{Execution}

The programming language we used to execute the code is Rust, it is a compiled,
general purpose, low-level programming language with a strong type system,
no garbage collector nor manual allocators but a lifetime borrow-checker that
makes code compiled with it memory safety, performant as C but with high-level
abstractions like with C++.

\subsection{Install Rust}

To install rust, on Unix-like systems only the following command is needed:

\begin{minted}{bash}
curl --proto '=https' --tlsv1.2 -sSf https://sh.rustup.rs | sh
\end{minted}

After the installation is finished, we must add \texttt{\$HOME/.cargo/bin} to the \texttt{\$PATH}.
Yo may want to restart your terminal emulator for the changes to the \texttt{\$PATH} to take effect.

\subsection{Install the program}

\subsubsection{Install from crates.io}

\texttt{crates.io} is the Rust community’s crate registry which allows users to publish and download \textit{crates}.
A crate is a compilation unit in Rust, like a library or an executable binary.

To install the program from \texttt{crates.io} use \texttt{cargo}, the Rust package manager.

\begin{minted}{bash}
cargo install aigraph1
\end{minted}

The command \texttt{aigraph1} should be available from the command prompt.
To execute the program run the command \texttt{aigraph1} in the terminal prompt.

\subsubsection{Install from GitHub}

The source code within the GitHub repository is the same the team sent you via the Moodle assignment.
We can either use \texttt{cargo} to compile from GitHub automatically or manually clone the repository and
compile.

To install using the source code from GitHub with a single command use:
\begin{minted}{bash}
cargo install --git https://github.com/AOx0/aigraph1
\end{minted}

Otherwise, to compile/install manually. First, clone the GitHub repository and move inside the created directory.

\begin{minted}{bash}
git clone https://github.com/AOx0/aigraph1 && cd ./airgraph1
\end{minted}

Now we can either install the program to \texttt{\$HOME/cargo/.bin} or run it in-place. When compiling a 
Rust program every temporary file and binary is placed inside a \texttt{target} directory. To clean everything
related to the project, just remove the target directory inside aigraph1.

Execute the following command to install the binary;

\begin{minted}{bash}
cargo install --path .
\end{minted}

The command \texttt{aigraph1} should be available from the command prompt.
To execute the program run the command \texttt{aigraph1} in the terminal prompt.

The alternative is to run in-place. For this purpose use the following command inside the \texttt{aigraph1} directory:

\begin{minted}{bash}
cargo run --release
\end{minted}

This command will compile and run the project.

\subsubsection{Install from source code}

The source code in the zip file is the same as in GitHub. To run or install the program unzip the file and move inside the
directory.


\begin{minted}{bash}
unzip -q aigrap1.zip && cd ./aigraph1
\end{minted}


Now we can either install the program to \texttt{\$HOME/cargo/.bin} or run it in-place. When compiling a 
Rust program every temporary file and binary is placed inside a \texttt{target} directory. To clean everything
related to the project, just remove the target directory inside aigraph1.

Execute the following command to install the binary;

\begin{minted}{bash}
cargo install --path .
\end{minted}

The command \texttt{aigraph1} should be available from the command prompt.
To execute the program run the command \texttt{aigraph1} in the terminal prompt.

The alternative is to run in-place. For this purpose use the following command inside the \texttt{aigraph1} directory:

\begin{minted}{bash}
cargo run --release
\end{minted}

This command will compile and run the project.


\newpage

\subsection{Descifrado}
\begin{minted}{rust}

pub struct Graph<I, N, E, Ty = Directed, Ix = DefaultIx> {
    pub inner: PGraph<N, E, Ty, Ix>,
    pub nodes: HashMap<I, NodeIndex<Ix>>,
}
\end{minted}

\newpage
\section{Conclusiones y Recomendaciones}


\newpage
\section{Bibliografía}
\printbibliography[heading=none]

\newpage

\appendix

\section{Appendix}

Comparación de histograma de las ocurrencias de bytes entre un arreglo de bytes
y su versión cifrada por el \texttt{Método 1}

\begin{figure}[H]
    \centering
    \includegraphics[scale=0.7]{../img/historygram}
    \caption*{Histograma de los bytes del archivo
original y el cifrado.}\label{fig:d2}
\end{figure}

Gráficas elaboradas en Mathematica 13 con el código:

donde \texttt{baa} es un arreglo de bytes conteniendo el estado del archivo
original y donde \texttt{bx} es una matriz que se usa como arreglo de bytes que
contiene el estado cifrado de los contenidos de donde \texttt{baa}.

\newpage
\section{Appendix}


\newpage
\section{Appendix: Notebook de Mathematica completo y repositorio de LaTeX}

\begin{enumerate}
\item El cuaderno está disponible en GitHub,\\ junto con todas las imágenes:
\texttt{https://github.com/AOx0/proyecto-al}
\item El repositorio de código de LaTeX para crear\\ este escrito está
disponible en\\  \texttt{https://www.overleaf.com/read/gkdvygxgntsh} y en
\texttt{https://github.com/AOx0/proyecto-al}
\end{enumerate}

